\documentclass[11pt]{article}
\usepackage{epsfig}    % to insert postscript figures
\parindent=0pt
\begin{document}

\title{Updating Euler approximations to ODE systems}
\author{ Hugh Murrell for Ben Murrell}
\maketitle

\begin{abstract}
The order in which variables are updated matters.
\end{abstract}

\section{Simulating the standard SIR model}
The continuous SIR model \cite{Wikipedia} 
\begin{eqnarray}
\frac{dS(t)}{dt} & = & - \beta \frac{S(t) I(t)}{N}  \nonumber \\
\frac{dI(t)}{dt} & = &  \beta \frac{S(t) I(t)}{N} - \gamma I(t)  )\label{eq2}  \\
\frac{dR(t)}{dt} & = &  \gamma I_t \nonumber
\end{eqnarray}
can be discretised using a Euler approximation as
\begin{eqnarray}
S_{t+\Delta t} & = & S_t - \Delta t \beta \frac{S_t I_t}{N}  \nonumber \\
I_{t+\Delta t} & = & I_t + \Delta t \beta \frac{S_t I_t}{N} -  \Delta t \gamma I_t  \label{eq2}  \\
R_{t+\Delta t} & = & R_t + \Delta t  \gamma I_t \nonumber
\end{eqnarray}
and this discrete system can be simulated correctly using the code in appendix A

\section{Updating before sampling}
Now consider the code in Appendix B where first {\tt new\_cases} are sampled
and then {\tt S} and {\tt I} are updated and then {\tt recoveries} are sampled and
then {\tt I} and {\tt R} are updated. With this code we are simulating the
following discrete system:
\begin{eqnarray}
S_{t+\Delta t} & = & S_t - \Delta t \beta \frac{S_t I_t}{N}  \nonumber \\
I_{t+\Delta t} & = & I_t + \Delta t \beta \frac{S_t I_t}{N} -  \Delta t \gamma ( I_t + \Delta t \beta \frac{S_t I_t}{N} )  \nonumber \\
                     & = &  I_t + \Delta t \beta \frac{S_t I_t}{N} -  \Delta t \gamma ( I_{t+\Delta t} + \Delta t \gamma I_t ) \label{eq3}  \\
R_{t+\Delta t} & = & R_t + \Delta t  \gamma ( I_t + \Delta t \beta \frac{S_t I_t}{N} ) \nonumber
\end{eqnarray}
\section{Conclusions}
The order in which sampling and updating occurs matters greatly

\newpage
\appendix
\section{}
\begin{verbatim}
  beta = R0 * gamma; 
  S = pop; I = init; R = 0;
  C = I;  
  for i in 1:upper
      # do the sampling before the updates
      new_cases = min(rand(Poisson(beta*S*I/(S+I+R))),S)
      recoveries = min(rand(Poisson(gamma*I)),I)
      # now do the updates
      I += new_cases
      S -= new_cases        
      I -= recoveries
      R += recoveries      
      C += new_cases  
      Rt = R0*(S/(S+I+R)) 
      # store the current counts in vectors 
      NC[i],Sv[i],Iv[i],Rv[i],Cv[i],Rtv[i] = new_cases,S,I,R,C,Rt
  end
\end{verbatim}

\section{}
\begin{verbatim}
  beta = R0 * gamma; 
  S = pop; I = init; R = 0;
  C = I;  
  for i in 1:upper
      # do some sampling and some updates
      new_cases = min(rand(Poisson(beta*S*I/(S+I+R))),S)
      I += new_cases
      S -= new_cases   
      # now do more sampling and the rest of the updates    
      recoveries = min(rand(Poisson(gamma*I)),I)
      I -= recoveries
      R += recoveries      
      C += new_cases  
      Rt = R0*(S/(S+I+R)) 
      # store the current counts in vectors 
      NC[i],Sv[i],Iv[i],Rv[i],Cv[i],Rtv[i] = new_cases,S,I,R,C,Rt
  end
\end{verbatim}

\newpage

\begin{thebibliography}{99}

\bibitem{Wikipedia} Wikipedia, {\em Compartmental models in epidemiology},
\verb+http://en.wikipedia.org/wiki/Compartmental_models_in_epidemiology+ .

\end{thebibliography}
\end{document}


